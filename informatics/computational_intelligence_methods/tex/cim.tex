\documentclass[12pt, letterpaper, twoside]{article}

\usepackage[utf8]{inputenc}
\usepackage{hyperref}
\usepackage{amsmath}

\hypersetup{
    colorlinks=true,
    linkcolor=color,
    filecolor=magenta,      
    urlcolor=blue,
}

\title{Computational Intelligence Methods}
\author{Jaroslav Langer \thanks{notes form lectures MIE-MVI/FIT/CTU}}
\date{Říjen 2020}



\begin{document}

\maketitle

\tableofcontents

\begin{abstract}
Definitions, terms and knowledge from course NI-MVI. \href{https://courses.fit.cvut.cz/MIE-MVI/}{Course page}.
\end{abstract}

\section{Lecture 1. Introduction}

\subsection*{}

\subsection*{What is intelligence}

\subsection*{Evolutionary}

\subsubsection*{Genotype fenotype}

\begin{itemize}
    \item fitness function
\end{itemize}

\subsection*{Significant fields}

\begin{itemize}
    \item self-driving cars
    \item intelligent assistents
    \item general artifical inteligence (play game from visual input)
\end{itemize}

\subsection*{Research at Datalab}

\subsection*{prg.ai}

\subsection*{ethics}

\section{Lecture 2. Machine Learning}

\subsection*{}

\subsection*{History}

\begin{itemize}
    \item 1940
    \item 1950
    \item 1960
    \item 1970
    \item 1980
    \item 1990
    \item 2000
    \item 2010
    \subitem GAN
    \subitem Transformers (pros of conv + recu)
    \item 2020+
    \subitem AutoML in RL
\end{itemize}

\subsection*{Machine learning tasks}

\begin{itemize}
    \item regression / prediction
    \item classification / recomendation
    \item clustering / 
    \item problem solving / planing / control 
\end{itemize}

\subsection{Types learning}

\begin{itemize}
    \item supervised
    \item unsupervised
    \item semisupervised
    \item Active learning
    \item transfer learning
    \item few-shot learning
    \item meta-learning / continual learning
\end{itemize}

\subsection*{}

\subsection*{Examples by types}

\subsection*{Measuring the performace}

\subsection*{Learning systems}

\subsection*{Defining learning task}

\begin{itemize}
    \item T - task ()
    \item P - performance
    \item E - expirience
\end{itemize}

\subsection*{Design learning system}

\begin{itemize}
    \item database, prepare data
    \item choose what to be learnt - target function
    \item choose representation of target function
    \item choose learning algoritm
    \item supply algorithm with performance metric
\end{itemize}

\subsection*{Checkers example}

Building the database
\begin{itemize}
    \item Direct expirience
    \subitem set of board with correct move
    \item indirect expirience
    \subitem sequences of game moves and final results
\end{itemize}

Choose target funciton
\begin{itemize}
    \item choseMove(board, legalMoves) -> bestMove
    \item V(board) -> R (how favorible position) - can be applied for all legalMoves
\end{itemize}

Choose target funciton representation

\subsection*{Machine learning methods}

\subsection*{Ants AI challenge}

\section{Lecture 3. Evolutionary Algos}

\subsection*{}


\section{Lecture 4. Neural Networks}

\subsection*{Overview}
\begin{itemize}
    \item Introduction to artificial neural networks
    \item Perceptron, gradient learning
    \item MLP, Back-propagation of error
    \item Unsupervized training - SOM
\end{itemize}

\subsection{Perceptron}

\subsubsection{Perceptron training}

\subsubsection{Perceptron gradient learning}

\subsubsection{Deriving gradietn of error}

\subsubsection{Cross entropy loss}


\subsection{Backpropagation algorithm}

\subsubsection{Multilayer perceptron – MLP}

\subsubsection{Chain rule and backprop}

\subsubsection{Training MLP}

\subsubsection{Propagating the error through multiple layers}

\subsubsection{Backprop summary}

\subsection{Backpropagation algorithm variants}

\subsubsection{Backprop variants}

\subsubsection{Modified transfer functions}

\subsubsection{Backprop with momentum}

\subsubsection{Batch updates and variable learning rate}

\subsubsection{Second order methods}

\subsection{MLP as universal approximator}

\subsection{Self-organizing Map}

\section{Convolutional Networks}



\end{document}
