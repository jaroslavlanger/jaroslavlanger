\documentclass[12pt, letterpaper, twoside]{article}

\usepackage[utf8]{inputenc}
\usepackage{hyperref}
\usepackage{amsmath}
\usepackage{amsfonts}

\hypersetup{
    colorlinks=true,
    linkcolor=color,
    filecolor=magenta,      
    urlcolor=blue,
}

\title{Algebra}
\author{Jaroslav Langer \thanks{z přednášek NI-MPI/FIT/ČVUT}}
\date{Říjen 2020}



\begin{document}

\maketitle

\tableofcontents

\begin{abstract}
Definice, pojmy a znalosti z předmětu NI-MPI. \href{https://courses.fit.cvut.cz/NI-MPI/lectures/index.html}{Odkaz na courses}.
\end{abstract}

\section{Přednáška 1.}

\subsection*{Grupoid, grupa, Cayleyho tabulka; podgrupa}

\subsection{Úvod}

\textbf{Věta 1.1}: Pro všechna $b,c \in R \setminus \emptyset$ pro rovnici $b \cdot x = c$ existuje právě jedno řešení $x = \frac{c}{b}$

\textbf{Poznámka}: Dvojici $(M, \circ), \textrm{kde} \quad \circ: M \times M \to M$,
$\circ$ je asociativní na M, neutrální prvek náleží M a pro každý prvek M inverzní prvek náleží M, říkáme grupa.

\subsection{Hierarchie}

\textbf{Definice 2.1}: \textbf{Grupoid (magma)} je uspořádaná dvojice $(M, \circ)$,
kde $M$ je neprázdná množina a $\circ$ je binární operace na $M$.
\begin{itemize}
    \item \textbf{Pologrupa} (semigrupa) je grupoid $(M, \circ)$,
        kde operace $\circ$ je asociativní pro všechny prvky $M$.
    \item \textbf{Monoid} je pologrupa $(M, \circ)$,
        kde $\exists e, \forall a, \quad e,a \in M,$ \[a \circ e = e \circ a = a\].
    \item \textbf{Grupa} je monoid $(M, \circ)$ a $e \in M$ je neutrální prvek,
        kde $\forall a \exists a^{-1}, \quad e,a \in M,$ \[a \circ a^{-1} = a^{-1} \circ a = e\].
    \item \textbf{Komutativní (abelovská) grupa} je grupa $(M, \circ)$,
        kde operace $\circ$ je komutativní na $M$.
\end{itemize}

\textbf{Poznámka}: O množině $M$ mluvíme také jako o \textit{nosiči} grupy $(M, \circ)$

\textbf{Poznámka}: Binární operace je \textit{na} $M$ což znamená, že $\circ: M \times M \to M$,
 také můžeme říci, že množina $M$ ja uzavřená vůči operaci $\circ$.

\subsection{Neutrální a inverzní prvky}

\textbf{Věta 4.1}: V monoidu existuje právě jeden neutrální prvek.

\textbf{Věta 4.2}: V grupě má každý prvek právě jeden inverzní prvek.

\subsection{Znázornění grup}

\textbf{Poznámka}: Pokud má množina M konečný počet prvků, pak strukturu dvojic $(M, \circ)$,
 lze kompletně zachytit Cayleyho tabulkou. Cayleyho tabulka pro $(M, \circ), |M|=n$
 je tabulka $n \times n$, kde záhlaví sloupců i řádků jsou stejně seřazené prvky $M$,
 políčko $p_{a,b}$ pro $a$-tý řádek a $b$-tý sloupec má hodnotu $a \circ b$

\textbf{Poznámka}: Latinský čtverec pro $n$ prvkovou množinu $M$ je tabulka $n \times n$,
 kde v každém řádku a sloupci, je každý prvek $M$ právě jednou.

\textbf{Věta 5.2}: Cayleyho tabulka každé grupy tvoří latinský čtverec.

\textbf{Věta 5.3}: V každé grupě $(M, \circ)$ jde jednoznačně dělit.
 Tzn. $\forall a,b \in M$ mají rovnice \[a \circ x = b, y \circ a = b\] jediné řešení.

\textbf{Poznámka}: Grupu $(M, \circ)$ s konečným počtem prvků $M$ lze vizualizovat pomocí textit{Cayleyho orientovaného grafu}.
 Cayleyho orientovaný graf \[(V,E), V=M, E=\{(a,b): b=a \circ c, \quad \forall a \in M, \forall c \in N \subset V\}\]

\section{Přednáška 2.}

\subsection*{dodělávka Algebry I; Algebra II : podgrupy; Lagrangeova věta; cyklické grupy}

\subsection{Podgrupy}

\textbf{Poznámka}: Hledáme-li podgrupu $(N, \circ)$ grupy $(M, \circ)$ tak aby obsahovala prvek $m$,
 těleso musí zůstat grupou, proto musí také obsahovat všechny prvky tak, aby množina $N$ byla uzavřená na operaci $\circ$,
 dále musí obsahovat neutrální prvek $e$, a inverzní prvek pro všechny prvky $N$.
 Takovou podgrupu nazýváme \textbf{podgrupa generovaná množinou $\{m\}$}.

\textbf{Definice 6.2}: \textbf{Podgrupa (subgroup) $(N, \circ)$}
 buď grupa $G = (M, \circ)$, podgrupa $H = (N, \circ)$ je libovolná dvojice, kde
\begin{itemize}
    \item $N \subset M$
    \item $(N, \circ)$ je grupa.
\end{itemize}

\textbf{Poznámka}: Každá grupa $(M, \circ)$, kde $|M| \geq 2$ má vždy podrgrupy
\begin{itemize}
    \item $(\{e\}, \circ), e \in M$
    \item $(N, \circ), N = M$
\end{itemize}
těmto dvěma podgrupám říkáme \textbf{triviální podgrupy}. Ostatní podgrupy nazýváme \textbf{valstní (proper)}.

\textbf{Věta 6.3}: Buď grupa $G = (M, \circ)$, pro každé $i$ z indexové množiny $I$ buď $H_{i}$ podgrupa $G$, pak 
\[H' = \bigcap_{i \in I} H_i\] je také podgrupa $G$.

\textbf{Věta 6.4}: Buď grupa $G = (M, \circ), N \subset M \wedge N \neq \emptyset$,
 pak libovolná dvojice $(N, \circ)$ je podgrupa právě tehdy když
\[ \forall a,b \in N, a \circ b^{-1} \in N \]

\subsection{Lagrangeova věta}

\textbf{Definice 7.1 Řád (order)}: \textbf{Řádem grupy} $G = (M, \circ)$ nazýváme počet prvků $M$,
 jeli počet prvků nekonečný, i řád je nekonečný, podle řádů rozdělujeme grupy na \textbf{konečné} a \textbf{nekonečné.}
 Řád grupy $G$ značíme $\#G$ (nebo také $|G| = ord(G)$).

\textbf{Věta 7.3 Lagrangeova}: Buď $H = (N, \circ)$ podgrupa konečné grupy $G = (M, \circ)$, potom řád $H$ dělí řád $G$.

\textbf{Věta 7.4 Sylowova}: Buď grupa konečná $G$ řádu $n$ a $p$ prvočíselný dělitel $n$. Pokud $p^k$ dělí $n$ (pro $k$ přírozená),
 potom existuje podgrupa $G$ řádu $p^k$. (Pro $k = 1$ též Cauchyho věta).

\subsection{Generující množiny a generátor grup}

\textbf{Věta 8.1} Buď grupa $G = (M, \circ)$ a $N \subset M \wedge N \neq \emptyset$, pak množina
\[ \langle N \rangle = \bigcap\{H: \textrm{$H$ je podgrupa grupy $G$ obsahující $N$}\} \]
spolu s operací $\circ$ tvoří podgrupu grupy $G$ obsahující prvek $N$.

\textbf{Věta 8.2} Podgrupu $\langle N \rangle$ grupy $G = (M, \circ), N \subset M \wedge N \neq \emptyset$
 nazýváme \textbf{podgrupou generovanou množinou $N$}.
 O množinu $N$ pak nazýváme jako \textbf{generující množinu} grupy $\langle N \rangle$.
V případě jednoprvkové generující množiny zavádíme značení $\langle a \rangle = \langle \{ a \} \rangle$
 nazýváme jednoprvkovou množinu \textbf{generátor} grupy $\langle a \rangle$.

\textbf{Poznámka}: Pro grupu $G = (M, \circ)$ s neutrálním prvkem $e \in M$
 pro každý prvek $g \in M$ a $n \in \mathbb{N}$ zavádíme $n$-tou a $-n$-tou mocninu takto.

\begin{align*}
    &g^0 = e \\
    &g^1 = g \\
    &g^2 = g \circ g \\
    &g^n = g \circ g \circ g ... \circ g \quad \textrm{($n$-krát)} \\
    &g^{-2} = g^{-1} \circ g^{-1} \\
    &g^{-n} = (g^{-1})^{n}
\end{align*}

\textbf{Věta 8.5} Buď grupa $G = (M, \circ)$ a podmnožina $N \subset M \wedge N \neq \emptyset$,
 potom všechny prvky grupy $\langle N \rangle$ lze získat pomocí grupového obalu
\[\langle N \rangle = \{a_{1}^{k_1} \circ a_{2}^{k_2} \circ ... a_{n}^{k_n} \circ: n \in \mathbb{N}, k_i \in \mathbb{Z}, a_i \in N\} \]

\textbf{Důsledek}: $ \langle N \rangle = \{a^k: k \in \mathbb{Z}\}$

\subsection{Cyklické grupy}

\textbf{Věta 9.1} Grupa $\mathbb{Z}^+_n$ je rovna $\langle k \rangle, k \in mathbb{Z}^+_n$
 tehdy a jen tehdy, když $k$ a $n$ jsou nesoudělná čísla.

\textbf{Definice 9.4 Cyklická grupa (cyclic group)} Grupa $G = (M, \circ)$ se nazývá \textbf{cyklická},
 pokud existuje $a \in M, \langle a \rangle = G$. Prvek $a$ se nazývá \textbf{generátor} cyklické grupy $G$.

\textbf{Definice 9.5} Buď $g$ prvek grupy $G = (M, \circ)$, existuje-li $m \in \mathbb{N}$ takové, že $g^m = e$,
 pak nejmenší takové $m$ nazýváme \textbf{řádem prvku g}. Neexistuje-li takové $m$, pak prvek $g$ má řád nekonečno.
 Řád prvku $g$ značíme $ord(g)$.

\textbf{Poznámka} Řád prvku $g$ se rovná řádu množiny generované $g$, platí tedy rovnost
\[ord(g) = \#\langle g \rangle\]
Dále platí, že $ g^k = e \Leftrightarrow k = l \cdot ord(g), l \in \mathbb{Z}$

\textbf{Věta 9.6} Grupa $\mathbb{Z}^{\times}_{n}$ je cyklická právě tehdy když $n \in \{2, 4, p^k, 2p^k\}, k \in \mathbb{N}$
 a $p$ je liché prvočíslo.

\textbf{Poznámka} Obecně najít generátor grupy není jednoduché, (třeba pro grupy $\mathbb{Z}^{\times}_{n}$).
 Pokud však jeden známe, je jednoduché najít všechny ostatní.

\textbf{Věta 9.7} Je-li $G = (M, \circ)$ cyklická grupa řádu $n$ a $a$ nějaký její generátor.
 Potom $a^k$ je také její tehdy a jen tehdy, když $n$ a $k$ jsou nesoudělné, tedy $gcd(n,k) = 1$

\textbf{!!! Důkaz !!! brutus}

\textbf{Poznámka} $\varphi(n)$ je \textbf{Eulerova funkce},
 každému $n \in \mathbb{N}$ přiřazuje počet menších přirozených čísel, která jsou s ním nesoudělná.

\textbf{Věta 9.8} V cyklické grupě řádu $n$ je počet generátorů roven $\varphi(n)$.

\textbf{Věta 9.9} Libovolná podgrupa cyklické grupy je opět cyklická grupa.

\subsection{(Malá) Fermatova věta}

\textbf{Věta 10.1} V grupě $G = (M, \circ)$ řádu $n$ pro libovolný prvek $a \in M$ platí
 $a^n = e$, kde $e$ je neutrální prvek. 

\textbf{Poznámka} Grupa $ \mathbb{Z}^{\times}_{p}$ je cyklická a řádu $p - 1$.

\textbf{Věta 10.2} Pro libovolné $p$ a libovolné $1 < a < p$
\[a^{p-1} \equiv 1 (mod \quad p). \quad (a^n \equiv a (mod \quad p))\] 

\section{Přednáška 3.}

\subsection{Homomorfismy a izomorfismy}

\textbf{Definice 11.1} Buď $G = (M, \circ_G), H = (N, \circ_H)$ dva grupoidy,
 zobrazení $h: M \to N$, nazveme \textbf{homomorfismem G do H}, jestliže
\[ \forall a,b \in M, h(a \circ_G b) = h(a) \circ_H h(b)\]
Je-li navíc $h$ injektivní, resp. surjektivní, resp. bijektivní, říkáme,
 že jde o \textbf{monomorfismus}, resp. \textbf{epimorfismus}, resp. \textbf{izomorfismus}.

\textbf{Definice 11.2} Grupy $G = (M, \circ_G), H = (N, \circ_H)$ nazýváme \textbf{izomorfní},
 právě tehdy když existuje izomorfismus $h: M \to N$, také říkáme, že G je izomorfní s H.

\textbf{Poznámka}: vlastnost dvou grup být izomorfní je relace ekvivalence na množině všech grup.

\textbf{Věta 11.3}: buď homomorfismus grupy $G = (M, \circ_G)$ do grupoidu $H = (N, \circ_H)$ $h: M \to N$,
potom $h(G) = (h(M), \circ_H) grupa$

\subsection{Aplikace teorie grup v kryptografii}

\section{Přednáška 4.}

\section{Přednáška 5.}

\end{document}
