\documentclass[12pt, letterpaper, twoside]{article}

\usepackage{fontspec}
\usepackage{hyperref}
\usepackage{amsmath}
\usepackage{amsfonts}

\hypersetup{
    colorlinks=true,
    linkcolor=color,
    filecolor=magenta,
    urlcolor=blue,
}

\title{Lineární algebra}
\author{Jaroslav Langer \thanks{přednášky BI-LIN/FIT/ČVUT}}
\date{Říjen 2020}



\begin{document}

\maketitle

\tableofcontents

\begin{abstract}
Definice, pojmy a znalosti z předmětu BI-LIN. \href{https://courses.fit.cvut.cz/BI-LIN/}{Courses předmětu}.
\end{abstract}

\section{Kapitola 1.}

\subsection*{Gaussova Eliminační metoda (GEM)}

\subsection*{Horní stupňovitý tvar}

\section{Kapitola 2.}
\subsection*{Základní pojmy lineární algebry}

\subsection{Lineární obal}
\textbf{Definice}: Buď $(x_{1},x_{2},\dots, x_{3})$ soubor vektorů z $V$. 
Množinu všech lineárních kombinací $(x_{1},x_{2},\dots, x_{3})$ nazveme \textbf{lineární obal souboru}.
Značíme ji
\[
    \langle x_{1},x_{2},\dots, x_{3} \rangle
\]
Buď $\emptyset \neq M \subset V$ množinu všech lineárních kombinací všech souborů vektorů z M nazýváme
\textbf{lineárním obalem množiny M} a značíme ji $\langle M \rangle$


\section{Kapitola 3.}
\subsection*{Hodnost matice a Frobeniova věta}

\subsection{Hodnost matice}

\textbf{Definice}: Hodností matice A nazýváme dimenzi lineárního obalu souboru řádků matice a značíme ji $h(A)$

\subsection{Regulární matice a maticová inverze}

\textbf{Definice}: \textbf{Kroneckerovo delta}

\textbf{Definice}: \textbf{Jednotková matice}

\textbf{Definice}: Buď $A \in T^{n,n}$. Existuje-li matice $B \in T^{n,n}$ taková, že
\[
    AB = BA = E
\] matice $A$ je \textbf{regulární} a matici $B$ nazveme \textbf{inverzní maticí} k matici $A$.
Značíme $B = A^{-1}$. Pokud matice $A$ není regulární, nazýváme ji \textbf{singulární.}.

\textbf{Věta}: Je-li matice $A \in T^{n,n}$ regulární, pak je inverzní matice k $A$ určena jednoznačně.

\textbf{Definice}: Nechť $V = T^n$ je libovolný.
\begin{itemize}
    \item Varietu o dimenzi 0 nazýváme bod.
    \item Varietu o dimenzi 1 nazýváme přímka.
    \item Varietu o dimenzi 2 nazýváme rovina.
    \item Varietu o kodimenzi 1 nazýváme nadrovina.
\end{itemize}

\section{Kapitola 4.}

\section{Kapitola 5.}

\subsection{Základní pojmy}

Zobrazení $f: X \to Y$ je \textbf{injektivní (prosté)}, pokud
\[\forall x,y \in X f(x) = f(y) \Rightarrow x = y\].

Zobrazení $f: X \to Y$ je \textbf{surjektivní (na)}, pokud
\[\forall y \in Y, \exists x \in X, f(x) = y\].

Zobrazení $f: X \to Y$ je \textbf{bijektivní (vzájemně jednoznačné)}, pokud
je současně injektivní i surjektivní.


\subsection{Hodnost, jádro a defekt zobrazení}
\subsubsection{Injektivita a surjektivita zobrazení}



\section{Kapitola 6.}
\subsection*{Determinant matice}

\subsection{Permutace}

\textbf{Definice 6.1}: Nechť $n \in \mathbb{N}, \hat{n} = \{1,2,...,n\}$ \textbf{permutací množiny $\hat{n}$}
 nazveme libovolné zobrazení $\pi: \hat{n} \to \hat{n}$, které je bijekce.

\subsubsection{Transpozice}

Nyní zmíníme speciální jednoduchý druh permutací.
Permutace, které odpovídají prohození právě dvou prvků v množině $\hat{n}$, budeme nazývat transpozice.

\textbf{Definice}: Nechť $n \in N$ a $i, j \in \hat{n}, i \neq j$. Permutaci $\tau_{ij} \in S_{n}$, kde

\begin{enumerate}
    \item $\tau_{ij} (j) = i$,
    \item $\tau_{ij} (i) = j$,
    \item $\tau_{ij} (k) = k$, pro k 6= i, j,
\end{enumerate}

nazýváme transpozicí čísel $i$ a $j$.

\subsection{Definice determinantu}

\textbf{Věta}: Matice $A \in T^{n,n}$ je regulární právě tehdy když $detA ̈́\neq 0$


\subsection*{Vlastní čísla}

\end{document}
