\documentclass[12pt, letterpaper, twoside]{article}

\usepackage[utf8]{inputenc}
\usepackage{hyperref}
\usepackage{amsmath}

\hypersetup{
    colorlinks=true,
    linkcolor=color,
    filecolor=magenta,      
    urlcolor=blue,
}

\title{Nelineární optimalizace a numerické metody}
\author{Jaroslav Langer \thanks{z přednášek NI-NON/FIT/ČVUT pana profesora Jaroslava Kruise}}
\date{Říjen 2020}



\begin{document}

\maketitle

\tableofcontents

\begin{abstract}
Definice, pojmy a znalosti z předmětu NI-NON. \href{http://mech.fsv.cvut.cz/~jk/minon2021.html}{Odkaz na prezentace}.
\end{abstract}

\section{Přednáška 1.}

\subsection*{Úvod, motivace, funkce, derivace funkce, extrémy funkcí, kvadratické funkce n proměnných, metoda největšího spádu}

\subsection{Funkce, extrémy funkcí}

\subsubsection{Derivace funkce}

\textbf{Definice}: Nechť je funkce f(x), limita

\begin{math}
    \frac{d f(x)}{d x} = \lim_{h \to 0} \frac{f(x+h)-f(x)}{h}
\end{math}

se nazývá derivace funkce f(x) v bodě x, někdy značeno $f'(x)$

\subsubsection{Derivace některých funkcí}

\begin{center}
\begin{tabular}{ |c|c|c| }
    \hline
    $f(x)$ & $f'(x)$ \\
    \hline
    $x^n$ & $nx^{(n-1)}$ \\
    $\sin x$ & $\cos x$ \\
    $\cos x$ & $- \sin x$ \\
    $n^x$ & $n^x \cdot \ln n$ \\
    $\log_{n} x$ & $ \frac{1}{x \ln n}$ \\
    \hline
\end{tabular}
\end{center}

\textbf{Definice}: Bod $x \in D$ se nazývá stacionárním právě když $f'(x) = 0$

\subsubsection{Konvexní a konkávní funkce}

\textbf{Definice}: Funkce $f(x)$ se nazývá konvexní na množině $M \subset D$, když $\forall x_{1},x_{2} \in D, \alpha \in (0,1)$ platí
\begin{math}
    f(\alpha x_{1} + (1-\alpha)x_{2}) \le \alpha f(x_{1}) + (1-\alpha)f(x_{2})
\end{math}

\textbf{Věta}: jeli $f'(x) > 0$ funkce je v bodě x rostoucí.

\textbf{Věta}: jeli $f'(x) < 0$ funkce je v bodě x klesající.

\textbf{Věta}: jeli $f''(x) > 0$ funkce je v bodě k ryze konvexní.

\textbf{Věta}: jeli $f''(x) < 0$ funkce je v bodě k ryze konkávní.

\textbf{Věta}: jeli $f'(x) = 0 \wedge f''(x) > 0$ funkce má v bodě x lokální minimum.

\textbf{Věta}: jeli $f'(x) = 0 \wedge f''(x) < 0$ funkce má v bodě x lokální maximum.

\subsection{Kvadratické programování}

\subsection{Kvadratická funkce mnoha proměnných}

\begin{multline}
    f(x) = \frac{1}{2} a_{1,1}x_{1}^{2} + \frac{1}{2} a_{1,2}x_{1}x_{2} + \dots + \frac{1}{2} a_{1,n}x_{1}x_{n} \\
    + \frac{1}{2} a_{2,1}x_{2}x_{1} + \frac{1}{2} a_{2,2}x_{2}^{2} + \dots + \frac{1}{2} a_{2,n}x_{2}x_{n} \\
    \dots \\
    + \frac{1}{2} a_{n,1}x_{n}x_{1} + \frac{1}{2} a_{n,2}x_{n}x_{2} + \dots + \frac{1}{2} a_{n,n}x_{n}^{2} \\
    + b_{1}x_{1} + b_{2}x_{2} + \dots + b_{n}x_{n} + c
\end{multline}

potom se dají proměnné $x_{1}, x_{2}, \dots x_{n} \textrm{napsat jako vektor} \quad \textbf{v}$

a kvadratická rovnice n proměnných se dá maticově napsat jako

\[
    f(x) = x^{T}Ax + x^{T}b + c
\]

$c$ nemění vlastnosti extrému, pouze jeho absolutní hodnotu, proto se dále předpokládá, že $c = 0$

\textbf{Definice}: Matice $A^{n,n}$ je pozitivně definitní, právě tehdy když pro libovolný nenulový vektor $x$ 
\[
    x \in R^{n}, x^TAx > 0
\]

\textbf{Poznámka}: Velmi mnoho inženýrských úloh se dá zkonstruovat tak, 
že se hledá minimum kvadratické rovnice o n neznámýchjako s pozitivně definitní maticí.

\textbf{Věta}: Matice $A^{n,n}$ je pozitivně definitní právě tehdy když jsou všechna její vlastní čísla kladná.

\textbf{def}: Matice $A$ je symetrická právě tehdy když $A = A^T$



% - inverzní matice se nikdy nepočítá, byla by plná, moc veklá (ostatní matice jsou řídké)

% - determinant a kubická rovnice -> vlastní čísla
% - tři rovnice -> valstní tři vektory

% \section{Přednáška 2.}

% \subsection*{přehled}

% \section{TODO}

% \begin{itemize}
%     \item vlatní číslo 
%     \item vlastní vektor
%     \item fundamentální matice
% \end{itemize}

\end{document}
