\documentclass[12pt, a4paper]{article} % 12pt font, a4paper size.
\usepackage{minted} % Necessary for code blocks.
\usepackage{hyperref} % Required for web links.
\usepackage{amsmath} % Needed for multiline math.
\usepackage[section]{placeins} % Keeps listings in the right sections.
\hypersetup{
     colorlinks=true, % Otherwise the links are in color boxes.
     linkcolor=black, % E.g. hyperlinks in table of contents.
     urlcolor=blue, % Classical blue web links.
     }
\title{How to \LaTeX}
\author{Jaroslav Langer}
% To specify a date use: \date{February 4, 2021}

\begin{document} % The document begins here.
\maketitle % Inserts a title with name and date.

\begin{abstract}
\noindent All the information one needs to use \href{https://www.latex-project.org/about/}{Latex} instead of \href{https://daringfireball.net/projects/markdown/}{Markdown}.
\end{abstract}

\tableofcontents % Inserts a table of contents.

\section{Getting Started} % \section is equivalent to heading 1.

\begin{listing}[h] % [h] is required for the figure to be "here", not floating.
    \caption{Latex installation.} % Text from \caption{} will be put under.
    \begin{minted}{shell}
    sudo apt install texlive-full
    \end{minted}
\end{listing}

\begin{listing}[h]
    \caption{First latex document.}
    \begin{minted}{latex}

    \documentclass{article}
    \begin{document}
    Hello \LaTeX!
    \end{document}

    \end{minted}
\end{listing}



\begin{listing}[h]
    \caption{Generating PDF from first.tex.}
    \begin{minted}{shell}
    pdflatex first.tex
    # If you use minted package, add -shell-escape flag.
    pdflatex -shell-escape first.tex
    \end{minted}
\end{listing}

\begin{itemize} % Unordered list.
    \item \textbf{For details see the source of this file.}
\end{itemize}

\subsection*{Mathematics}

Inline $\sum^{n}_{i}$ math example.\par
\bigskip % Inserts an empty line.
\noindent Multiline math example:

\begin{multline} \\
    8x - 3y = 5\\
    2x + 5y = 7\\
\end{multline}

\noindent Number for each equation:

\begin{align}
    \mathbf{x}(t) = \mathbf{A}(t)\mathbf{x}(t) + \mathbf{B}(t)\mathbf{u}(t)\\
    \mathbf{y}(t) = \mathbf{C}(t)\mathbf{x}(t) + \mathbf{D}(t)\mathbf{u}(t)
\end{align}

\subsection*{Non-ASCII Characters} % The star hides it from the contents.

Est-il possible d'utiliser des caractères français? \par
% To create a new paragraph either use blank line or put \par at the end.
\noindent A co takhle psaní v češtině?
% Every paragraph (except the first one) is indented without \noindent.

\subsection*{Tables}

\begin{center}
\begin{tabular}{ |c|c|c| }
    \hline
    x & o & x\\
    \hline
    o & x & o\\
    \hline
    x & x & o\\
    \hline

\end{tabular}
\end{center}

\subsection*{Code}

\begin{listing}[!h]
    \caption{Python code example.}
    \begin{minted}[linenos=true, breaklines=true]{python}
    from datetime import datetime

    def show_time(formats: list[str]) -> None:
        """Prints current time in all given formats."""
        for format_ in formats:
            assert(isinstance(format_, str)) # Just for sure.
            print(datetime.now().strftime(format_))
    \end{minted}
\end{listing}

\section{References}

\begin{enumerate}
    \item \url{https://www.overleaf.com/learn/latex/Learn_LaTeX_in_30_minutes}
\end{enumerate}

\end{document}
