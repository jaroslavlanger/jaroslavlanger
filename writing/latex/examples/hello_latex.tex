\documentclass[12pt, letterpaper, twoside]{article}

\usepackage[utf8]{inputenc}
\usepackage{hyperref}

\hypersetup{
    colorlinks=true,
    linkcolor=color,
    filecolor=magenta,      
    urlcolor=blue,
}

\title{Hello \LaTeX{}}
\author{Jaroslav Langer \thanks{inspired by the Overleaf team}} 
\date{October 2020}

\begin{document}

\maketitle

\tableofcontents

% Just learning it

\begin{abstract}
Document to learn latex basics. So far it looks good.\href{https://www.overleaf.com/learn/latex/Learn_Latex_in_30_minutes}{Learn latex in 30 minutes - Overleaf}
\end{abstract}

% \chapter{Basics}

\section{Introduction}

Todays most of us start using computer for basic tasks. The text one are usually solved with Microsoft Word or similar software. This approach is called "What You See Is What You Get" (WYSIWYG). Once a person starts with programming, his perspectives changes. Using plain text files starts to be a daily routine and also meny documents starts to be in plain text. This naturally brings him to formats like markdown. Simple method of styling text with simple symbols. Such as \# for Headline 1, \#\# for Headline 2 etc. Markdown works great, until the formating become heavy. In that case it is time for something more powerful. To me latex seems just like it. 

\section{First touches}

Using lists is vivid for me in writing easy texts

\begin{itemize}
    \item It helps me to simplify the message.
    \item Also the reader can easily remember it.
\end{itemize}

\subsection{beyond the markdown}

Actually i start learning it msotly because of the abiity to use math $ \sum^{n}_{i}  $

\section*{So what is missing?}

Nothing?

\end{document}
